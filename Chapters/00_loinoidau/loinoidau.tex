%!TEX root = ../../book_.tex
\chapter{Lời nói đầu }
Chào các bạn,

Tài liệu này là một phần trong \href{https://machinelearningcoban.com/ebook}{Dự án viết ebook Machine Learning cơ bản} của tôi đang được thực hiện, dự kiến sẽ được hoàn thành cuối năm 2017. Khi chuẩn bị tài liệu này, blog đã có 32 bài viết và nhiều ghi chú ngắn về Machine Learning/Artificial Intelligence và Optimization. Hướng tiếp cận của tôi là giới thiệu mỗi thuật toán Machine Learning thông qua việc xây dựng một hàm số đặc biệt được gọi là \textit{hàm mất mát}, hoặc \textit{hàm mục tiêu} và các phương pháp tối ưu hàm mục tiêu. Để có thể hiểu sâu về các thuật toán Machine Learning, tôi luôn cho rằng hiểu rõ cách xây dựng hàm mất mát và cách tối ưu các hàm đó đóng một vai trò quan trọng. Vì vậy, tôi cũng dành thời gian cho một số bài viết liên quan đến Tối Ưu. 

Tài liệu này hiện gồm ba Phần trong cuốn ebook: Tối Ưu Lồi, Support Vector Machines và Ôn tập Đại Số Tuyến Tính. Trong đó, hai phần đầu tiên năm ở phần cuối cùng của cuốn ebook, là phần có nhiều kiến thức toán nhất. Phần thứ ba ở Phụ Lục được thêm vào để bổ sung những kiến thức toán quan trọng.

Trong lĩnh vực Tối Ưu, Tối Ưu Lồi đóng vai trò quan trọng hơn cả vì những tính chất quan trọng của nó. Tôi chọn ba bài viết về Tối Ưu Lồi để chuẩn bị cho tài liệu này vì tôi biết rằng nhiều bạn đọc trong blog muốn đọc những phần có nhiều toán trên giấy hơn là đọc trên máy. Việc in trực tiếp từ màn hình website ra không sự tốt vì dù sao việc chuyển đổi cũng là tự động. Không chỉ trong Machine Learning, các lĩnh vực khoa học kỹ thuật và cả tài chính kinh tế cũng rất cần tới tối ưu. Tôi hy vọng rằng tài liệu này sẽ giúp ích cho nhiều người Việt đang học tập và nghiên cứu ở trong và ngoài nước. 

Support Vector Machine là một trong những thuật toán đẹp nhất của Machine Learning. Bài toán tối ưu của nó được chứng mình là một bài toán lồi và nghiệm tìm được là duy nhất. Các biến thể khác của Machine Learning cũng được đề cập. 

Ngôn ngữ trong tài liệu này gần giống với ngôn ngữ trong blog và có liên quan chặt chẽ tới các bài viết khác mà tôi có dẫn links. Tuy nhiên, độc giả chưa đọc blog cũng có thể hiểu được vì đây là kiến thức tổng quan về Tối Ưu Lồi. 

% Nội dung của tài liệu được dựa trên cuốn \textbf{Convex Optimization} của tác giả nổi tiếng Stephen P. Boyd. Tôi cố gắng lược bỏ những phần đi quá sâu vào toán, đồng thời cũng thêm các ví dụ gần với thực tế và chương trình học toán ở Việt Nam. Các hình vẽ trong tài liệu đã được vẽ lại hoàn toàn. 

Tài liệu này được tổng hợp trong hai ngày, nội dung gần như tương tự như trên blog. Tuy nhiên, vì việc chuyển từ ngôn ngữ trên web sang LaTeX khá phức tạp nên chắc chắn tôi không tránh khỏi sai sót. Nếu thấy phần nào cần phải sửa lại, bạn hãy cho tôi biết qua địa chỉ email vuhuutiep@gmail.com. Tôi sẽ trả lời và chỉnh sửa ngay khi có thể. 


\textbf{Vấn đề bản quyền:}

Toàn bộ nội dung trong bài, source code và hình ảnh minh họa (trừ nội dung có trích dẫn) đều thuộc bản quyền của tôi, Vũ Hữu Tiệp. 

Tôi rất mong muốn kiến thức tôi viết trong blog này đến được với nhiều người. Tuy nhiên, tôi không ủng hộ bất kỳ một hình thức sao chép không trích nguồn nào. Mọi nguồn tin trích đăng bài viết cần nêu rõ tên blog (Machine Learning cơ bản), tên tác giả (Vũ Hữu Tiệp) và kèm link gốc của bài viết. Các bài viết trích dẫn quá 25\% toàn văn bất kỳ một post nào trong blog này là không được phép, trừ trường hợp có sự đồng ý của tác giả.

Mọi vấn đề liên quan đến việc sao chép, đăng tải, sử dụng bài viết, cũng như trao đổi, cộng tác, xin vui lòng liên hệ với tôi tại địa chỉ email: vuhuutiep@gmail.com.

Tôi xin chân thành cảm ơn!

Trân trọng,

Vũ Hữu Tiệp

\href{www.machinelearningcoban.com}{www.machinelearningcoban.com}

Hoa Kỳ, ngày 20 tháng 8 năm 2017.

