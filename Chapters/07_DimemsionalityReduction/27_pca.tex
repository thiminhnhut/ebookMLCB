%!TEX root = ../../book_ML.tex
\chapter{Phân tích thành phần chính}
\label{cha:pca}
% \index{principal component analysis}
% \index{PCA -- \textit{xem} principle component analysis}
\index{PCA}
 
% Và như thường lệ, tôi sẽ trình bày một phương pháp đơn giản nhất trong các thuật toán Dimensionality Reduction dựa trên một mô hình tuyến tính. Phương pháp này có tên là \textit{Principal Component Analysis} (PCA), tức \textit{Phân tích thành phần chính}. Phương pháp này dựa trên quan sát rằng dữ liệu thường không phân bố ngẫu nhiên trong không gian mà thường phân bố gần các đường/mặt đặc biệt nào đó. PCA xem xét một trường hợp đặc biệt khi các mặt đặc biệt đó có dạng tuyến tính là các không gian con (subspace). 
  
% Trước khi đi vào chi tiết của PCA, chúng ta cùng điểm lại một chút về Đại số tuyến tính và Thống kê. 
 
 
% \section{Một chút toán}
 
 
% \subsection{Norm 2 của ma trận}
% Chúng ta vẫn thường nhắc nhiều đến \href{http://machinelearningcoban.com/math/#-norms-chuan}{norm cho vector} nhưng chưa thực sự làm việc nhiều với norm của ma trận (ngoài \href{http://machinelearningcoban.com/math/#chuan-cua-ma-tran}{Frobenius norm}). Trong mục này, chúng ta sẽ làm quen với 1 lớp các norm cho ma trận được định nghĩa dựa trên norm của vector. Lớp các norms này còn được gọi là \textit{Induced Norms}. 
 
% Giả sử hàm số $\|\mathbf{x}\|_{\alpha}$ là một norm bất kỳ của vector $\mathbf{x}$. Ứng với norm này, định nghĩa norm tương ứng cho ma trận $\mathbf{A}$: 
% \begin{equation} 
% \|\mathbf{A}\|_{\alpha} = \max_{\mathbf{x}} \frac{\|\mathbf{Ax}\|_{\alpha}}{\|\mathbf{x}\|_{\alpha}} 
% \end{equation} 
 
% chú ý rằng ma trận $\mathbf{A}$ có thể không vuông và số cột của nó bằng với số chiều của $\mathbf{x}$. Như vậy, bản thân việc tính toán norm của ma trận là việc giải một bài toán tối ưu. Chú ý rằng hàm tối ưu có cả tử số và mẫu số là các norm trên vectors. 
 
% Chúng ta sẽ quan tâm nhiều hơn tới norm 2. Norm 2 của ma trận được định nghĩa là: 
% \begin{equation} 
% \label{eqn:27_1}
% \|\mathbf{A}\|_2 = \max_{\mathbf{x}} \frac{\|\mathbf{Ax}\|_2}{\|\mathbf{x}\|_2}
% \end{equation} 
 
% Nhận thấy rằng nếu $\mathbf{x}$ là nghiệm của bài toán tối ưu \eqref{eqn:27_1} thì $k\mathbf{x}$ cũng là nghiệm với $k$ là một số thực khác không bất kỳ. Không mất tính tổng quát, ta có thể giả sử mẫu số bằng 1. Khi đó, bài toán tối ưu \eqref{eqn:27_1} có thể được viết dưới dạng: 
% \begin{equation} 
%     \label{eqn:27_2}
%     \|\mathbf{A}\|_2 = \max_{\|\mathbf{x}\|_2 = 1} \|\mathbf{Ax}\|_2
% \end{equation} 
% Nói cách khác, ta cần đi tìm $\mathbf{x}$ sao cho: 
% \begin{equation}
% \label{eqn:27_3} 
% \begin{aligned} 
% \mathbf{x} &= \text{argmax}_{\mathbf{x}} \|\mathbf{Ax}\|_2^2  \\\ 
% \text{thoả mãn: } & \|\mathbf{x}\|_2^2 = 1  
% \end{aligned} 
% \end{equation} 
% Ở đây, các norm 2 đã được bình phương lên để tránh dấu căn bậc hai. Bài toán \eqref{eqn:27_3} có thể được giải bằng \href{http://machinelearningcoban.com/2017/04/02/duality/# -- phuong-phap-nhan-tu-lagrange}{Phương pháp nhân tử Lagrange} vì ràng buộc là một phương trình. 
 
% Lagrangian của Bài toán \eqref{eqn:27_3} là: 
% \begin{equation} 
% \mathcal{L}(\mathbf{x}, \lambda) = \|\mathbf{Ax}\|_2^2 + \lambda (1 - \|\mathbf{x}\|_2^2) 
% \end{equation} 
 
% Nghiệm của bài toán \eqref{eqn:27_3} sẽ thoả mãn hệ phương trình: 
% \begin{eqnarray} 
%     \label{eqn:27_4}
%     \frac{\partial \mathcal{L}}{\partial \mathbf{x}} &=& 2\mathbf{A}^T\mathbf{Ax} - 2\lambda \mathbf{x} = \mathbf{0} \\\ 
%     \label{eqn:27_5}
%     \frac{\partial \mathcal{L}}{\partial \lambda} &=& 1 - \|\mathbf{x}\|_2^2 = 0  
% \end{eqnarray} 
 
% Từ \eqref{eqn:27_4} ta có: 
% \begin{equation} 
%     \label{eqn:27_6}
%     \mathbf{A}^T\mathbf{Ax} = \lambda\mathbf{x}  
% \end{equation} 
% Điều này suy ra rằng $\lambda$ là một trị riêng của $\mathbf{A}^T\mathbf{A}$ và $\mathbf{x}$ là 1 vector riêng ứng với trị riêng đó. Tiếp tục nhân hai vế của \eqref{eqn:27_6} với $\mathbf{x}^T$ vào bên trái, ta có: 
% \begin{equation} 
% \mathbf{x}^T\mathbf{A}^T\mathbf{Ax} = \lambda \mathbf{x}^T\mathbf{x} = \lambda 
% \end{equation} 
% Nhận thấy rằng vế trái chính là $\|\mathbf{Ax}\|_2^2$ chính là hàm mục tiêu trong \eqref{eqn:27_3}. Vậy hàm mục tiêu đạt giá trị lớn nhất khi $\lambda$ đạt giá trị lớn nhất. Nói cách khác, $\lambda$ chính là trị riêng lớn nhất của $\mathbf{A}^T\mathbf{A}$ hay chính là \href{http://machinelearningcoban.com/2017/06/07/svd/#-nguon-goc-ten-goi-singular-value-decomposition}{singular value} lớn nhất của ma trận $\mathbf{A}$. 
 
% Như vậy, norm 2 của một ma trận chính là singular value lớn nhất của ma trận đó. Và nghiệm của bài toán \eqref{eqn:27_3} chính một là \textit{right-singular vector} ứng với singular value đó. 
 
% Với lý luận tương tự, chúng ta có thể suy ra rằng bài toán: 
% \begin{equation} 
%     \min_{\|\mathbf{x}\|_2 =1} \mathbf{x}^T\mathbf{A}^T\mathbf{A}\mathbf{x} 
% \end{equation} 
 
% có nghiệm là vector riêng ứng với trị riêng nhỏ nhất của $\mathbf{A}^T\mathbf{A}$. Khi đó, hàm số đạt giá trị nhỏ nhất bằng chính trị riêng nhỏ nhất này. 
 
 
 
 
 
% \subsection{Biễu diễn vector trong các hệ cơ sở khác nhau}
 
% Trong không gian $D$ chiều , toạ độ của mỗi điểm được xác định dựa trên một hệ toạ độ nào đó. Ở các hệ toạ độ khác nhau, hiển nhiên là toạ độ của mỗi điểm cũng khác nhau. 
 
% Tập hợp các vector $\mathbf{e}_1, \dots, \mathbf{e}_D$ mà mỗi vector $\mathbf{e}_d$ có đúng 1 phần tử khác 0 ở thành phần thứ $d$ và phần tử đó bằng 1, được gọi là hệ cơ sở đơn vị (hoặc hệ đơn vị) trong không gian $D$ chiều. Nếu xếp các vector $\mathbf{e}_d, d = 1, 2, \dots, D$ theo đúng thứ tự đó, ta sẽ được ma trận đơn vị $D$ chiều. 
 
% Mỗi vector cột $\mathbf{x} = [x_1, x_2, \dots, x_D] \in \mathbb{R}^D$, biểu diễn của nó trong hệ đơn vị là: 
 
% \begin{equation} 
%     \mathbf{x} = x_1 \mathbf{e}_1 + x_2 \mathbf{e}_2 + \dots + x_D\mathbf{e}_D 
% \end{equation} 
 
% Giả sử có một hệ cơ sở khác $\mathbf{u}_1, \mathbf{u}_2, \dots, \mathbf{u}_D$ (các vector này độc lập tuyến tính), vậy thì biểu diễn của vector $\mathbf{x}$ trong hệ cơ sở mới này có dạng: 
 
% \begin{equation} 
%     \mathbf{x} = y_1 \mathbf{u}_1 + y_2 \mathbf{u}_2 + \dots + y_D\mathbf{u}_D  = \bU\mathbf{y} 
% \end{equation} 
 
% $\bU$ là ma trận mà cột thứ $d$ của nó chính là vector $\mathbf{u}_d$. Lúc này, vector $\mathbf{y}$ chính là biểu diễn của $\mathbf{x}$ trong hệ cơ sở mới. Bộ các số $y_d, d = 1, 2, \dots, D$ là duy nhất vì $\mathbf{y}$ có thể tính được bằng: 
% \begin{equation} 
%     \label{eqn:27_7}
%     \mathbf{y} = \bU^{-1} \mathbf{x} 
% \end{equation} 
% với chú ý rằng $\bU$ là ma trận khả nghịch vì các cột của nó độc lập tuyến tính. 
 
% Trong các ma trận đóng vai trò như hệ cơ sở $\bU$, \href{http://machinelearningcoban.com/2017/06/07/svd/#-he-truc-giao-va-truc-chuan}{các ma trận trực giao}, tức $\bU^T\bU = \mathbf{I}$, được quan tâm nhiều hơn vì nghịch đảo của chúng chính là chuyển vị của chúng: 
% \begin{equation} 
%     \bU^{-1} = \bU^T 
% \end{equation} 
% Khi đó, $\mathbf{y}$ trong \eqref{eqn:27_7} có thể được tính một cách nhanh chóng: 
 
% \begin{equation} 
%     \mathbf{y} = \bU^{T} \mathbf{x} 
% \end{equation} 
% từ đó suy ra: $y_i = \mathbf{x}^T \mathbf{u}_i = \mathbf{u}_i^T\mathbf{x}, i= 1, 2, \dots, D$. 
 
% Có thể nhận thấy rằng vector $\mathbf{0}$ được biểu diễn như nhau trong mọi hệ cơ sở. 
% Hình 1 dưới đây là 1 ví dụ về việc chuyển hệ cơ sở: 
 
% % <hr> 
% % <div> 
% % <table width = "100%" style = "border: 0px solid white"> 
 
% %     <tr > 
% %         <td width="40%" style = "border: 0px solid white" align = "center"> 
% %         <img style="display:block;" width = "100%" src = "/assets/27_pca/changebasis.png"> 
% %          </td> 
% %         <td width="40%" style = "border: 0px solid white" align = "justify"> 
% %         Hình 1: Chuyển đổi toạ độ trong các hệ cơ sở khác nhau. 
 
% %         </td> 
% %     </tr> 
% % </table> 
% % </div> 
% % <hr> 
% \begin{figure}[t]
%     % caption on side     
%     \floatbox[{\capbeside\thisfloatsetup{capbesideposition={right,top},capbesidewidth=6cm}}]{figure}[\FBwidth]
%     {\caption{ 
%     Chuyển đổi toạ độ trong các hệ cơ sở khác nhau.
%     }
%     \label{fig:27_1}}
%     { % figure here
%     \includegraphics[width=.5\textwidth]{Chapters/07_DimemsionalityReduction/27_pca/latex/changebasis.pdf}
%     }
% \end{figure}

 
% Việc chuyển đổi hệ cơ sở sử dụng ma trận trực giao có thể được coi như một phép xoay trục toạ độ. Nhìn theo một cách khác, đây cũng chính là một phép xoay vector dữ liệu theo chiều ngược lại. 
% \subsection{Trace}
 
% Hàm số trace xác định trên tập các ma trận vuông được sử dụng rất nhiều trong tối ưu vì những tính chất đẹp của nó. Hàm trace trả về tổng các phần tử trên đường chéo của một ma trận vuông. 
 
% Các tính chất quan trọng của hàm trace, với giả sử rằng các ma trận trong hàm trace là vuông và các phép nhân ma trận thực hiện được: 
% \begin{itemize}
%     \item $\text{trace}(\mathbf{A}) = \text{trace}(\mathbf{A}^T)$ 
     
%     \item $\text{trace}(k\mathbf{A}) = k\text{trace}(\mathbf{A})$ với $k$ là một số bất kỳ. 
     
%     \item $\text{trace}(\mathbf{AB}) = \text{trace}(\mathbf{BA})$ 
     
%     \item $\|\mathbf{A}\|_F^2 = \text{trace}(\mathbf{A}^T\mathbf{A}) = \text{trace}(\mathbf{A}\mathbf{A}^T)$ với $\mathbf{A}$ là ma trận bất kỳ, có thể không vuông. 
     
%     \item $\text{trace}(\mathbf{A}) = \sum_{i = 1}^D \lambda_i $ với $\mathbf{A}$ là một ma trận vuông và $\lambda_i, i = 1, 2, \dots, N$ là toàn bộ các trị riêng của nó, có thể phức hoặc lặp. Việc chứng minh tính chất này có thể được dựa trên ma trận đặc trưng của $\mathbf{A}$ và định lý Viète. Tôi xin được bỏ qua. 
% \end{itemize}
 
 
% \subsection{Kỳ vọng và ma trận hiệp phương sai}
 
% \subsubsection{Dữ liệu một chiều}
 
% Cho $N$ giá trị $x_1, x_2, \dots, x_N$. \textit{Kỳ vọng} và \textit{phương sai} của bộ dữ liệu này được định nghĩa là: 
 
% \begin{eqnarray} 
%     \bar{x} &=& \frac{1}{N}\sum_{n=1}^N x_n = \frac{1}{N}\mathbf{X1}\\\ 
%     \sigma^2 &=& \frac{1}{N} \sum_{n=1}^N (x_n - \bar{x})^2 
% \end{eqnarray} 
% với $\mathbf{1} \in \mathbb{R}^N$ là vector cột chứa toàn phần tử 1. 
% Kỳ vọng đơn giản là trung bình cộng của toàn bộ các giá trị. Phương sai là trung bình cộng của bình phương khoảng cách từ mỗi điểm tới kỳ vọng. Phương sai càng nhỏ thì các điểm dữ liệu càng gần với kỳ vọng, tức các điểm dữ liệu càng giống nhau. Phương sai càng lớn thì ta nói dữ liệu càng có tính phân tán. Ví dụ về kỳ vọng và phương sai của dữ liệu một chiều có thể được thấy trong Hình 2a). 
 
% Căn bậc hai của phương sai, $\sigma$ còn được gọi là độ lệch chuẩn (standard deviation) của dữ liệu. 
 
 
% \subsubsection{Dữ liệu nhiều chiều}
 
% Cho $N$ điểm dữ liệu được biểu diễn bởi các vector cột $\mathbf{x}_1, \dots, \mathbf{x}_N$, khi đó, \textit{vector kỳ vọng} và \textit{ma trận hiệp phương sai} của toàn bộ dữ liệu được định nghĩa là: 
 
% \begin{eqnarray} 
%     \bar{\mathbf{x}} &=& \frac{1}{N} \sum_{n=1}^N \mathbf{x}_n \\\ 
%     \mathbf{S} &=&  \frac{1}{N}\sum_{n=1}^N (\mathbf{x}_n - \bar{\mathbf{x}})(\mathbf{x}_n - \bar{\mathbf{x}})^T = \frac{1}{N}\what{\bX}\what{\bX}^T 
% \end{eqnarray} 
% Trong đó $\what{\bX}$ được tạo bằng cách trừ mỗi cột của $\bX$ đi $\bar{\mathbf{x}}$: 
% \begin{equation} 
%     \hat{\mathbf{x}}_n = \mathbf{x}_n - \bar{\mathbf{x}} 
% \end{equation} 
 
% Các công thức này khá tương đồng với các công thức cho dữ liệu 1 chiều phía trên. Có một vài điểm lưu ý: 
% \begin{itemize}
%     \item Ma trận hiệp phương sai là một ma trận đối xứng, hơn nữa, nó là một ma trận \href{https://machinelearningcoban.com/2017/03/12/convexity/#positive-semidefinite}{nửa xác định dương}. 
     
%     \item Mọi phần tử trên đường chéo của ma trận hiệp phương sai là các số không âm. Chúng cũng chính là phương sai của từng chiều của dữ liệu. 
     
%     \item Các phần tử ngoài đường chéo $s_{ij}, i \neq j$ thể hiện sự tương quan giữa thành phần thứ $i$ và thứ $j$ của dữ liệu, còn được gọi là hiệp phương sai. Giá trị này có thể dương, âm hoặc bằng 0. Khi nó bằng 0, ta nói rằng hai thành phần $i, j$ trong dữ liệu là không tương quan (uncorrelated). 
     
%     \item Nếu ma trận hiệp phương sai là ma trận đường chéo, ta có dữ liệu hoàn toàn không tương quan giữa các chiều. 
% \end{itemize}
 
% Ví dụ về dữ liệu không tương quan và tương quan được cho trong Hình 2bc). 
% % <hr> 
% % <div> 
% % <table width = "100%" style = "border: 0px solid white"> 
% %     <tr > 
% %         <td width="40%" style = "border: 0px solid white" align = "center"> 
% %         <img style="display:block;" width = "100%" src = "/assets/27_pca/var_1d.png"> 
% %          <br> 
% %         a) 
% %          </td> 
% %          <td width="40%" style = "border: 0px solid white" align = "center"> 
% %          <img style="display:block;" width = "100%" src = "/assets/27_pca/pca_diagvar.png"> 
% %           <br> 
% %          b) 
% %           </td> 
% %     </tr> 
 
% %     <tr > 
% %         <td width="40%" style = "border: 0px solid white" align = "center"> 
% %         <img style="display:block;" width = "100%" src = "/assets/27_pca/pca_var0.png"> 
% %          <br> 
% %         c) 
% %          </td> 
% %          <td width="40%" style = "border: 0px solid white" align = "justify"> 
% %         Hình 2: Ví dụ về kỳ vọng và phương sai. a) Trong không gian 1 chiều. b) Không gian 2 chiều mà hai chiều không tương quan. Trong trường hợp này, ma trận hiệp phương sai là ma trận đường chéo với hai phần tử trên đường chéo  là $\sigma_1, \sigma_2$, đây cũng chính là hai trị riêng của ma trận hiệp phương sai và là phương sai của mỗi chiều dữ liệu. c) Dữ liệu trong không gian hai chiều có tương quan. Theo mỗi chiều, ta có thể tính được kỳ vọng và phương sai. Phương sai càng lớn thì dữ liệu trong chiều đó càng phân tán. Trong ví dụ này, dữ liệu theo chiều thứ hai phân tán nhiều hơn so so với chiều thứ nhất. 
% %           </td> 
% %     </tr> 
% % </table> 
% % </div> 
% % <hr> 
 
% %%%%%%% Three subfigures with bottom caption%%%%%%%%%%%%%%
%  \begin{figure}[t]
%      \begin{subfigure}{0.329\textwidth}
%      \includegraphics[width=0.99\linewidth]{Chapters/07_DimemsionalityReduction/27_pca/latex/var_1d.pdf}
%      \caption{}
%      % \label{fig:subim1}
%      \end{subfigure}
%      \begin{subfigure}{0.329\textwidth}
%      \includegraphics[width=0.99\linewidth]{Chapters/07_DimemsionalityReduction/27_pca/latex/pca_diagvar.pdf}
%      \caption{}
%      % \label{fig:subim2}
%      \end{subfigure}
%      \begin{subfigure}{0.329\textwidth}
%      \includegraphics[width=0.99\linewidth]{Chapters/07_DimemsionalityReduction/27_pca/latex/pca_var0.pdf}
%      \caption{}
%      % \label{fig:subim2}
%      \end{subfigure}
%      \caption{
%      Ví dụ về kỳ vọng và phương sai. a) Trong không gian 1 chiều. b) Không gian 2 chiều mà hai chiều không tương quan. Trong trường hợp này, ma trận hiệp phương sai là ma trận đường chéo với hai phần tử trên đường chéo  là $\sigma_1, \sigma_2$, đây cũng chính là hai trị riêng của ma trận hiệp phương sai và là phương sai của mỗi chiều dữ liệu. c) Dữ liệu trong không gian hai chiều có tương quan. Theo mỗi chiều, ta có thể tính được kỳ vọng và phương sai. Phương sai càng lớn thì dữ liệu trong chiều đó càng phân tán. Trong ví dụ này, dữ liệu theo chiều thứ hai phân tán nhiều hơn so so với chiều thứ nhất. 
%      }
%      \label{fig:27_2}
%  \end{figure}
  
\index{phân tích thành phần chính -- principle component analysis}
\index{PCA}
\section{Phân tích thành phần chính}
\subsection{Ý tưởng} % (fold)


% subsection ý_tưởng (end)
%%%%%%% Three subfigures with bottom caption%%%%%%%%%%%%%%
 \begin{figure}[t]
     \begin{subfigure}{0.59\textwidth}
     \includegraphics[width=0.99\linewidth]{Chapters/07_DimemsionalityReduction/27_pca/latex/pca_diagvar.pdf}
     \caption{}
     \label{fig:pca_2a}
     \end{subfigure}
     \begin{subfigure}{0.33\textwidth}
     \includegraphics[width=0.99\linewidth]{Chapters/07_DimemsionalityReduction/27_pca/latex/pca_var0.pdf}
     \caption{}
     \label{fig:pca_2b}
     \end{subfigure}
     \caption{Ví dụ về phương sai của dữ liệu trong không gian hai chiều. (a)
     Phương sai của chiều thử hai (tỉ lệ với độ rộng của đường hình chuông) nhỏ
     hơn phương sai của chiều thứ nhất. (b) Cả hai chiều có phương sai đáng kể. Phương sai của
     mỗi chiều là phương sai của thành phần tương ứng được lấy trên toàn bộ dữ
     liệu. Phương sai tỉ lệ thuận với độ phân tán của dữ liệu.}
     \label{fig:pca_2}
 \end{figure}

Giả sử vector dữ liệu ban đầu $\bx \in \R^{D}$ được giảm chiều trở thành $\bz
\in \R^K$ với $K < D$. Một cách đơn giản để giảm chiều dữ liệu từ $D$ về $K < D$ là chỉ giữ lại $K$ phần tử {quan trọng nhất}. Có hai câu hỏi lập tức
được đặt ra. Thứ nhất, làm thế nào để xác định {tầm quan trọng} của
mỗi phần tử? Thứ hai, nếu tầm quan trọng của các phần tử là như
nhau, ta cần bỏ đi những phần tử nào?

Để trả lời câu hỏi thứ nhất, hãy quan sát Hình~\ref{fig:pca_2a}. Giả sử các
điểm dữ liệu có thành phần thứ hai (phương đứng) giống hệt nhau hoặc sai khác
nhau không đáng kể (phương sai nhỏ). Khi đó, thành phần này hoàn toàn có thể được lược
bỏ và ta ngầm hiểu rằng nó sẽ được xấp xỉ bằng kỳ vọng của thành phần đó
trên toàn bộ dữ liệu. Ngược lại, nếu áp dụng phương pháp này lên chiều thứ nhất (phương ngang), {lượng thông tin} bị mất đi đáng kể do sai số xấp xỉ quá lớn. Vì vậy, lượng thông tin theo mỗi thành phần có thể được đo bằng phương sai của dữ liệu trên thành phần đó. Tổng lượng
thông tin là tổng phương sai trên toàn bộ các thành phần. Lấy
một ví dụ về việc có hai camera được đặt dùng để chụp cùng một người, một camera
phía trước và một camera đặt trên đầu. Rõ ràng, hình ảnh thu được từ
camera đặt phía trước mang nhiều thông tin hơn so với hình ảnh nhìn từ
phía trên đầu. Vì vậy, bức ảnh chụp từ phía trên đầu có thể được bỏ qua mà không
làm mất đi quá nhiều thông tin về hình dáng của người đó.

Câu hỏi thứ hai tương ứng với trường hợp Hình~\ref{fig:pca_2b}. Trong cả hai
chiều, phương sai của dữ liệu đều lớn; việc bỏ đi một trong hai chiều đều khiến
một lượng thông tin đáng kể bị mất đi. Tuy nhiên, nếu xoay trục toạ độ đi một
góc phù hợp, một trong hai chiều dữ liệu có thể được lược bở vì dữ liệu có xu
hướng phân bố xung quanh một đường thẳng.
 
\textit{Phân tích thành phần chính} (principle component analysis, PCA) là
phương pháp đi tìm một phép xoay trục toạ độ để được một hệ trục toạ độ mới sao
cho trong hệ mới này, thông tin của dữ liệu chủ yếu tập trung ở một vài thành
phần. Phần còn lại chứa ít thông tin hơn có thể được lược bỏ.

Phép xoay trục toạ độ có liên hệ chặt chẽ tới hệ trực chuẩn và ma trận trực giao
(xem Mục~\ref{sec:linalg_orthogonality} và~\ref{sec:doi_he_co_so}). Giả sử hệ cơ
sở trực chuẩn mới là $\bU$ (mỗi cột của $\bU$ là một vector đơn vị cho
một chiều) và ta muốn giữ lại $K$ toạ độ trong hệ
cơ sở mới này. Không mất tính tổng quát, giả sử đó là $K$ thành phần đầu tiên.
% ******************************************************************************
\begin{figure}[t]
\centering
    \includegraphics[width = \textwidth]{Chapters/07_DimemsionalityReduction/27_pca/latex/pca_idea.pdf}
    \caption[]{Ý tưởng chính của PCA: Tìm một hệ trực chuẩn mới sao cho trong hệ này, các thành phần quan trọng nhất nằm trong $K$ thành phần đầu tiên.}
    \label{fig:27_3}
\end{figure}
% ******************************************************************************
Quan sát Hình~\ref{fig:27_3} với cơ sở mới $\bU =
[\bU_K, \what{\bU}_K]$ là một hệ trực chuẩn với $\bU_K$ là ma trận con tạo bởi $K$ cột đầu tiên của $\bU$. Trong hệ cơ sở mới này, ma trận dữ liệu có thể được viết thành
\begin{equation} 
\label{eqn:27_8}
    \bX = \bU_K \mathbf{Z} + \widehat{\bU}_K \mathbf{Y}
\end{equation} 
Từ đây ta cũng suy ra
\begin{eqnarray} 
\label{eqn:27_9}
    \left[\begin{matrix} \mathbf{Z} \\\ \mathbf{Y} \end{matrix} \right] =  
    \left[\begin{matrix} \bU_K^T \\\ \what{\bU}_K^T \end{matrix} \right]\bX \Rightarrow 
    \begin{matrix} 
    \mathbf{Z} = \bU_K^T \bX \\\ 
    \mathbf{Y} = \what{\bU}_K^T\bX 
    \end{matrix} 
\end{eqnarray}
Mục đích của PCA là đi tìm ma trận trực giao $\bU$ sao cho phần lớn thông tin
nằm ở $\bU_K \mathbf{Z}$, phần nhỏ thông tin nằm ở $\what{\bU}_K\mathbf{Y}$.
Phần nhỏ này sẽ được lược bỏ và xấp xỉ bằng một ma trận có các
cột như nhau. Gọi mỗi cột đó là
$\mathbf{b}$, khi đó, ta sẽ xấp xỉ $\mathbf{Y} \approx
\mathbf{b1}^T$ với $\mathbf{1}^T\in \mathbb{R}^{1
\times N}$ là một vector hàng có toàn
bộ các phần tử bằng một. Giả sử đã tìm được $\bU$, ta cần tìm $\mathbf{b}$ thoả mãn:
\begin{equation} 
    \mathbf{b} = \text{argmin}_{\bb} \|\mathbf{Y} - \mathbf{b1}^T\|_F^2 =
    \text{argmin}_{\mathbf{b}} \|\what{\bU}_K^T\bX - \mathbf{b1}^T\|_F^2 
\end{equation} 
Giải phương trình đạo hàm theo $\mathbf{b}$ của hàm mục tiêu bằng $\bzero$: 
\begin{equation} 
    (\mathbf{b1}^T - \what{\bU}_K^T\bX)\mathbf{1} = 0 \Rightarrow N\mathbf{b} = \what{\bU}_K^T \mathbf{X1} \Rightarrow \mathbf{b} = \what{\bU}_K^T \lbar{\mathbf{x}}.
\end{equation} 
Ở đây ta đã sử dụng $\bone^T\bone = N$ và $\lbar{\bx} = \frac{1}{N}\bX\bone$ là
vector trung bình các cột của $\bX$. 
Với giá trị $\mathbf{b}$ tìm được này, dữ liệu ban đầu sẽ được xấp xỉ bởi
\begin{equation} 
\label{eqn:27_10}
\bX = \bU_K \bZ + \what{\bU}_k\bY \approx \bU_K\bZ + \what{\bU}_k \bb\bone^T
= \bU_K \mathbf{Z} + \what{\bU}_K \what{\bU}_K^T\bar{\mathbf{x}}\mathbf{1}^T
\triangleq \tilde{\bX}
    % \bX \approx \tilde{\bX} = \bU_K \mathbf{Z} + \what{\bU}_K \what{\bU}_K^T\bar{\mathbf{x}}\mathbf{1}^T 
\end{equation} 
\subsection{Hàm mất mát}
Hàm mất mát của PCA được coi như sai số của phép xấp xỉ, được định
nghĩa là 
\begin{eqnarray}
\nonumber
\frac{1}{N}\|\bX - \tilde{\bX}\|_F^2 &=& 
\frac{1}{N}\|\what{\bU}_K \bY-  \what{\bU}_K
\what{\bU}_K^T\lbar{\bx}\bone^T\|_F^2 = 
\frac{1}{N}\|\what{\bU}_K \what{\bU}_K^T \bX -  \what{\bU}_K
\what{\bU}_K^T \bar{\mathbf{x}}\mathbf{1}^T\|_F^2\\
\label{eqn:27_11}
&=& \frac{1}{N} \|\what{\bU}_k \what{\bU}_k^T(\bX - \lbar{\bx}\bone^T)\|_F^2
\triangleq J(\bU)
\end{eqnarray}
Chú ý rằng, nếu các cột của một ma trận $\mathbf{V}$ tạo thành một hệ
trực chuẩn thì với một ma trận $\mathbf{W}$ bất kỳ, ta luôn có
\begin{equation} 
    \|\mathbf{VW}\|_F^2 = \text{trace} (\mathbf{W}^T\mathbf{V}^T\mathbf{V} \mathbf{W}) = \text{trace}(\mathbf{W}^T\mathbf{W}) = \|\mathbf{W}\|_F^2 
\end{equation} 
Đặt $\what{\bX} = \bX - \lbar{\bx}\bone^T$. Ma trận này có được bằng cách trừ
mỗi cột của $\bX$ đi trung bình các cột của nó. Ta gọi $\what{\bX}$ là {ma trận dữ liệu đã
được chuẩn hoá}. Có thể thấy $\hat{\mathbf{x}}_n = \mathbf{x}_n -
\bar{\mathbf{x}},~\forall n = 1, 2, \dots, N$.

% Ta tạm gọi
% $\what{\bX}$ là ma trận dữ liệu chuẩn hoá. 

Vì vậy hàm mất mát trong~\eqref{eqn:27_11} có thể được viết lại thành: 
\begin{eqnarray} 
    J(\bU) &=&  \frac{1}{N} \|\what{\bU}_K^T\what{\bX} \|_F^2 = \frac{1}{N}
    \|\what{\bX}^T \what{\bU}_K \|_F^2 = 
    \frac{1}{N}\sum_{i = K+1}^D \|\what{\bX}^T\mathbf{u}_i \|_2^2 \\\ 
    \label{eqn:27_12}
    &=& \frac{1}{N} \sum_{i=K+1}^D \mathbf{u}_i^T\what{\bX}\what{\bX}^T \mathbf{u}_i 
    = \sum_{i=K+1}^D \mathbf{u}_i^T\mathbf{S} \mathbf{u}_i
\end{eqnarray} 
với $\mathbf{S} = \frac{1}{N}\what{\bX}\what{\bX}^T$ là ma trận hiệp phương sai
của dữ liệu và luôn là một ma trận nửa xác định dương (xem
Mục~\ref{sub:expectaion_covariance}).
 
Công việc còn lại là tìm các $\mathbf{u}_i$ để mất mát là nhỏ nhất. 

 Với ma trận $\bU$ trực giao bất kỳ, thay $K = 0$ vào \eqref{eqn:27_12} ta có
\begin{eqnarray} 
    L &=& \sum_{i=1}^D \mathbf{u}_i^T\mathbf{Su}_i = \frac{1}{N} \|\what{\bX}^T\bU\|_F^2 =\frac{1}{N} \text{trace}(\what{\bX}^T\bU \bU^T \what{\bX})  \\\ 
    \label{eqn:27_13}
    &=& \frac{1}{N} \text{trace} (\what{\bX}^T \what{\bX})  = \frac{1}{N} \text{trace} (\what{\bX} \what{\bX}^T) =\text{trace} (\mathbf{S}) = \sum_{i=1}^D \lambda_i
\end{eqnarray} 
Với $\lambda_1 \geq \lambda_2 \geq \dots \geq \lambda_D \geq 0$ là các trị riêng
của ma trận nửa xác định dương $\mathbf{S}$. Chú ý rằng các trị riêng này là
thực và không âm\footnote{Tổng các trị riêng của một ma trận vuông bất kỳ luôn
bằng vết của ma trận đó.}.
 
 
\textit{Như vậy $L$ không phụ thuộc vào cách chọn ma trận trực giao $\bU$} và
bằng tổng các phần tử trên đường chéo của $\mathbf{S}$. Nói cách khác, $L$ chính
là tổng các phương sai theo từng thành phần của dữ liệu ban
đầu\footnote{Mỗi thành phần trên đường chéo chính của ma trận hiệp phương sai
chính là phương sai của thành phần dữ liệu tương ứng.}.
 
Vì vậy, việc tối thiểu hàm mất mát $J$ được cho bởi \eqref{eqn:27_12} tương
đương với việc tối đa biểu thức 
\begin{equation} 
    F  = L - J = \sum_{i=1}^K \mathbf{u}_i \mathbf{S} \mathbf{u}_i^T 
\end{equation} 
\subsection{Tối ưu hàm mất mát}
Nghiệm của bài toán tối ưu hàm mất mát PCA được tìm dựa trên khẳng định sau
đây:
\newnote{}{
Nếu $\bS$ là một ma trận nửa xác định dương, bài toán tối ưu 
\begin{eqnarray}
    \max_{\bU_K} \sum_{i=1}^K \bu_i^T\bS\bu_i \\
    \text{thoả mãn:}~ \bU_K^T\bU_K = \bI 
\end{eqnarray}
có nghiệm $\bu_1, \dots, \bu_K$ là các vector riêng ứng với $K$ trị riêng (kể
cả lặp) lớn
nhất của $\bS$. Khi đó, giá trị lớn nhất của hàm mục tiêu là $\sum_{i=1}^K\lambda_i$, với
$\lambda_1 \geq \lambda_2 \geq \dots \geq \lambda_D$ là các trị riêng của $\bS$.
}

% \textbf{Định lý 1:} $F$ đạt giá trị lớn nhất bằng $\sum_{i=1}^K \lambda_i$ khi $\mathbf{u}_i$ là các vector riêng có norm 2 bằng 1 ứng với các trị riêng này. Tất nhiên, chúng ta không quên điều kiện trực giao giữa các $\mathbf{u}_i$. 
 
% Chú ý rằng $\lambda_i, i = 1, \dots, K$ chính là $K$ trị riêng lớn nhất của ma trận hiệp phương sai $\mathbf{S}$.
Khẳng định này có thể được chứng minh bằng quy nạp\footnote{Xin được bỏ qua
phần chứng minh. Bạn đọc có thể xem Excercise 12.1 trong tài liệu tham
khảo~\cite{bishop2006pattern} với lời giải tại \url{https://goo.gl/sM32pB}.}.

Trị riêng lớn nhất $\lambda_1$ của ma trận hiệp phương sai $\bS$ còn được gọi
là \textit{thành
phần chính thứ nhất} ({the first principal component}), trị riêng thứ hai
$\lambda_2$ được gọi là \textit{thành phần chính thứ hai},... Tên gọi
\textit{phân tích thành phần chính} ({principal component analysis}) bắt
nguồn từ đây. Ta chỉ giữ lại $K$ thành phần chính đầu tiên khi giảm chiều dữ
liệu dùng PCA. 

% ******************************************************************************
\begin{figure}[t]
    % caption on side     
   
   
    \floatbox[{\capbeside\thisfloatsetup{capbesideposition={right,top},capbesidewidth=7.5cm}}]{figure}[\FBwidth]
    {\caption{ PCA có thể được coi là phương pháp đi tìm một hệ cơ sở trực chuẩn
    đóng vai trò một phép xoay, sao cho trong hệ cơ sở mới này, phương sai theo
    một số chiều nào đó là không đáng kể và có thể lược bỏ. Trong hệ cơ sở ban đầu
    $\mathbf{O}\be_1\be_2$, phương sai theo mỗi chiều (độ rộng của các đường
    hình chuông nét liền) đều lớn. Trong không gian mới với hệ cơ sở
    $\mathbf{O}\bu_1\bu_2$, phương sai theo hai chiều (độ rộng của các đường
    hình chuông nét đứt) chênh lệch nhau đáng kể. Chiều dữ liệu có phương sai nhỏ
    có thể được lược bỏ vì dữ liệu theo chiều này ít phân tán. }
    \label{fig:27_4}}
    { % figure here
   
    \includegraphics[width=.4\textwidth]{Chapters/07_DimemsionalityReduction/27_pca/latex/pca_var.pdf}
    }
\end{figure}
% ******************************************************************************
Hình~\ref{fig:27_4} minh hoạ các thành phần chính với dữ liệu hai chiều.
Trong không gian ban đầu với các vector cơ sở $\mathbf{e}_1,
\mathbf{e}_2$, phương sai theo mỗi chiều dữ liệu (tỉ lệ với độ rộng của các hình chuông
nét liền) đều lớn. Trong hệ cơ sở mới $\mathbf{O}\mathbf{u}_1\mathbf{u}_2$,
phương sai theo chiều thứ hai $\hat{\sigma}_2^2$ nhỏ so với
$\hat{\sigma}_1^2$. Điều này chỉ ra rằng khi chiếu dữ liệu lên $\mathbf{u}_2$, ta
được các điểm rất gần nhau và gần với giá trị trung bình theo chiều đó. Trong
trường hợp này, vì giá trị trung bình theo mọi chiều bằng 0, ta có thể thay thế
toạ độ theo chiều $\mathbf{u}_2$ bằng 0. Rõ ràng là nếu dữ liệu có phương sai
càng nhỏ theo một chiều nào đó thì khi xấp xỉ chiều đó bằng một hằng số, sai số
xấp xỉ càng nhỏ. PCA thực chất là đi tìm một phép xoay tương ứng với một ma trận
trực giao sao cho trong hệ toạ độ mới, tồn tại các chiều có phương sai nhỏ
có thể được bỏ qua; ta chỉ cần giữ lại các chiều/thành phần khác quan trọng hơn. Như
đã khẳng định ở trên, tổng phương sai theo toàn bộ các chiều chiều trong một hệ cơ sở bất kỳ là
như nhau và bằng tổng các trị riêng của ma trận hiệp phương sai. Vì vậy, PCA còn
được coi là phương pháp giảm số chiều dữ liệu sao tổng phương sai còn lại là lớn
nhất.
 
 
% Tôi sẽ bỏ qua phần chứng minh của Định lý 1. Tuy nhiên, cũng nêu một vài ý để bạn đọc có thể hình dung: 
 
% Khi $K = 1$. Ta cần giải bài toán: 
% \begin{eqnarray} 
%     \max_{\mathbf{u}_1} &\mathbf{u}_1^T\mathbf{S} \mathbf{u}_1 \\\ 
%     \text{thoả mãn:} &\|\mathbf{u}_1\|_2 = 1 
% \end{eqnarray} 
 
% Như đã đề cập ở phía trên, hàm mục tiêu đạt giá trị lớn nhất bằng $\lambda_1$ khi $\mathbf{u}_1$ là một vector riêng của ma trận hiệp phương sai $\mathbf{S}$ tương ứng với trị riêng $\lambda_1$. Vậy định lý đúng với $K = 1$ 
 
% Giả sử $\mathbf{u}_1$ đã là vector riêng ứng với trị riêng lớn nhất của $\mathbf{S}$ thế thì nghiệm $\mathbf{u}_2$ của bài toán tối ưu: 
% \begin{equation} 
%     \label{27_21}
%     \begin{aligned}
%     \max_{\mathbf{u}_2} &\mathbf{u}_2^T\mathbf{S} \mathbf{u}_2 \\\ 
%     \text{thoả mãn:}~ &\|\mathbf{u}_2\|_2 = 1\\\ 
%     & \mathbf{u}_2^T \mathbf{u}_1 = 0 & 
%     \end{aligned}
% \end{equation} 
% là một vector riêng của $\mathbf{S}$ ứng với trị riêng lớn thứ hai $\lambda_2$ của nó. Chú ý rằng $\lambda_2$ có thể bằng $\lambda_1$ nếu không gian riêng ứng với $\lambda_1$ có số rank lớn hơn 1. 
 
% Nhận định này có thể được chứng minh bằng phương pháp nhân tử Lagrange. Thật vậy, Lagrangian của bài toán $(21)$ là: 
% \begin{equation} 
% \mathcal{L}( \mathbf{u}_2, \nu_1, \nu_2) = \mathbf{u}_2^T\mathbf{S} \mathbf{u}_2 + \nu_1\mathbf{u}_1^T\mathbf{u}_2 + \nu_2(1 - \mathbf{u}_2^T\mathbf{u}_2) 
% \end{equation} 
 
% Ta cần giải hệ phương trình đạo hàm của $\mathcal{L}$ theo từng biến bằng 0: 
% \begin{eqnarray} 
%     \label{eqn:27_22}
%     \frac{\partial \mathcal{L}}{\partial \mathbf{u}_2} &=& 2 \mathbf{Su}_2 + \nu_1 \mathbf{u}_1 - 2\nu_2\mathbf{u}_2 = 0\\\ 
%     \label{eqn:27_23}
%     \frac{\partial \mathcal{L}}{\partial \nu_1} &=& \mathbf{u}_1^T \mathbf{u}_2 = 0 \\\ 
%     \label{eqn:27_24}
%     \frac{\partial \mathcal{L}}{\partial \nu_2} &=& 1 - \mathbf{u}_2^T \mathbf{u}_2 = 0 \\\ 
% \end{eqnarray} 
 
 
% Nhân cả hai vế của \eqref{eqn:27_22} với $\mathbf{u}_1^T$ vào bên trái ta có: 
% \begin{equation} 
% 2\mathbf{u}_1^T\mathbf{Su}_2 + \nu_1 = 0 
% \end{equation} 
% Vì $\mathbf{Su}_1 = \lambda_1 \mathbf{u}_1 \Rightarrow \mathbf{u}_1^T\mathbf{Su}_2 = \lambda_1 \mathbf{u}_1^T\mathbf{u}_2 = 0$. Từ đó suy ra $\nu_1 = 0$ và \eqref{eqn:27_22} lúc này tương đương với: 
% \begin{equation} 
% \mathbf{Su}_2 = \nu_2\mathbf{u}_2 \Rightarrow \mathbf{u}_2^T\mathbf{S} \mathbf{u}_2 = \nu_2 
% \end{equation} 
% Vậy $\mathbf{u}_2$ là một vector riêng của $\mathbf{S}$ ứng với $\nu_2$. Và để hàm mục tiêu đạt giá trị lớn nhất, $\nu_2$ cần càng lớn càng tốt. Điều này dẫn đến $\nu_2$ phải là trị riêng thứ hai của $\mathbf{S}$. 
 
% Lập luận tương tự, ta có thể chứng minh được: Nếu $\mathbf{u}_i, i = 1, 2, \dots, k-1$ là các vector riêng ứng với trị riêng lớn thứ $i$ của ma trận nửa xác định dương $\mathbf{S}$, hơn nữa, $k-1$ vector riêng này tạo thành một hệ trực chuẩn, thế thì: 
 
% \begin{eqnarray} 
%     \max_{\mathbf{u}_k} & \mathbf{u}_k^T\mathbf{Su}_k \\\ 
%     \text{thoả mãn:}~ & \mathbf{u}_k^T\mathbf{u}_k = 1; \\\ 
%     & \mathbf{u}_k^T\mathbf{u}_i = 1, i = 1,\dots, k -1 
% \end{eqnarray} 
% bằng đúng với trị riêng tiếp theo $\lambda_k$ tại $\mathbf{u}_k$ là vector riêng ứng với trị riêng này. 
 
 
 
\section{Các bước thực hiện phân tích thành phần chính}
Từ các suy luận trên, ta có thể tóm tắt lại các bước trong PCA như sau: 
\begin{itemize}
    \item[1)] Tính vector trung bình của toàn bộ dữ liệu:
       \begin{math} 
       \bar{\mathbf{x}} = \frac{1}{N} \sum_{n=1}^N \mathbf{x}_n 
       \end{math}.
    \item[2)] Trừ mỗi điểm dữ liệu đi vector trung bình của toàn bộ dữ liệu để được dữ
    liệu chuẩn hoá:
       \begin{equation} 
       \hat{\mathbf{x}}_n = \mathbf{x}_n - \bar{\mathbf{x}} 
       \end{equation} 
    \item[3)] Đặt $\what{\bX} = [\what{\bx}_1, \what{\bx}_2, \dots,
    \what{\bx}_D]$ là ma trận dữ liệu chuẩn hoá, tính ma trận hiệp phương sai:
       \begin{equation} 
       \mathbf{S} = \frac{1}{N}\what{\bX}\what{\bX}^T 
       \end{equation} 
    \item[4)] Tính các trị riêng và vector riêng tương ứng có $\ell_2$ norm bằng 1
    của ma trận này, sắp xếp chúng theo thứ tự giảm dần của trị riêng.
    \item[5)] Chọn $K$ vector riêng ứng với $K$ trị riêng lớn nhất để xây dựng ma trận $\bU_K$ có các cột tạo thành một hệ trực giao. $K$ vector này được gọi là các thành phần chính, tạo thành một không gian con {gần} với phân bố của dữ liệu ban đầu đã chuẩn hoá. 
    \item[6)] Chiếu dữ liệu ban đầu đã chuẩn hoá $\what{\bX}$ xuống không gian con tìm được. 
    \item[7)] Dữ liệu mới là toạ độ của các điểm dữ liệu trên không gian mới: 
       \begin{math} 
       \mathbf{Z} = \bU_K^T\what{\bX} 
       \end{math}.
\end{itemize}

\textit{Như vậy, PCA là kết hợp của phép tịnh tiến, xoay trục toạ độ và chiếu dữ liệu lên hệ toạ độ mới.} 

Dữ liệu ban đầu có thể tính được xấp xỉ theo dữ liệu mới bởi 
\begin{math} 
    \mathbf{x} \approx \bU_K\mathbf{Z} + \bar{\mathbf{x}} 
\end{math}.

Một điểm dữ liệu mới $\bv \in \R^D$ sẽ
được giảm chiều bằng PCA theo công thức $\bw = \bU_K^T(\bv - \lbar{\bx}) \in
\R^K$. Ngược lại, nếu biết $\bw$, ta có thể xấp xỉ $\bv$ bởi $\bU_K\bw +
\lbar{\bx}$. Các bước thực hiện PCA được minh hoạ trong Hình \ref{fig:27_5}.
 
% ******************************************************************************
\begin{figure}[t]
\centering
    \includegraphics[width = \textwidth]{Chapters/07_DimemsionalityReduction/27_pca/latex/pca_procedure.pdf}
    \caption[]{Các bước thực hiện PCA.}
    \label{fig:27_5}
\end{figure}
% ******************************************************************************

\section{Liên hệ với phân tích giá trị suy biến}
PCA và SVD có mối quan hệ đặc biệt với nhau. Xin phép nhắc lại hai điểm đã trình bày dưới đây: 
 
 
\subsection{SVD cho bài toán xấp xỉ hạng thấp tốt nhất}
Nghiệm của bài toán xấp xỉ một ma trận bởi một ma trận có hạng không vượt quá $k$ (xem Chương~\ref{cha:svd}: 
\begin{equation} 
\label{eqn:28_1}
\begin{aligned} 
  \min_{\mathbf{A}} &\|\bX - \mathbf{A}\|_F \\\ 
  \text{thoả mãn:}~ & \text{rank}(\mathbf{A}) = K 
\end{aligned} 
\end{equation} 
chính là SVD cắt ngọn của $\mathbf{A}$. 

Cụ thể, nếu SVD của $\bX
\in\mathbb{R}^{D\times N}$ là
\begin{equation} 
  \bX = \bU\mathbf{\Sigma}\mathbf{V}^T 
\end{equation} 
với $\bU \in \mathbb{R}^{D \times D}$ và $\mathbf{V}\in \mathbb{R}^{N\times N}$
là các ma trận trực giao và $\mathbf{\Sigma} \in \mathbb{R}^{D \times N}$ là ma
trận đường chéo (không nhất thiết vuông) với các phần tử trên đường chéo không
âm giảm dần. Nghiệm của bài toán \eqref{eqn:28_1} chính là:
\begin{equation} 
  \label{eqn:28_2}
  \mathbf{A} = \bU_K \mathbf{\Sigma}_K \mathbf{V}_K^T
\end{equation} 
với $\bU \in \mathbb{R}^{D \times K}$ và $\mathbf{V}\in \mathbb{R}^{N\times K}$
là các ma trận tạo bởi $K$ cột đầu tiên của $\bU$ và $\mathbf{V}$, 
$\mathbf{\Sigma}_K \in \mathbb{R}^{K \times K}$ là ma trận đường chéo con ứng
với $K$ hàng đầu tiên và $K$ cột đầu tiên của $\mathbf{\Sigma}$.
 
 
\subsection{Ý tưởng của PCA}
Như đã chứng minh ở \eqref{eqn:27_10}, PCA là bài toán đi tìm ma trận
trực giao $\bU$ và ma trận mô tả dữ liệu ở không gian thấp chiều $\mathbf{Z}$
sao cho việc xấp xỉ sau đây là tốt nhất:
\begin{equation} 
\label{eqn:28_3}
\bX \approx \tilde{\bX} = \bU_K \mathbf{Z} + \what{\bU}_K \what{\bU}_K^T\bar{\mathbf{x}}\mathbf{1}^T 
\end{equation} 
với $\bU_K, \what{\bU}_K$ lần lượt là các ma trận được tạo bởi $K$ cột đầu tiên
và $D-K$ cột cuối cùng của ma trận trực giao $\bU$ và $\bar{\mathbf{x}}$ là
vector trung bình của dữ liệu.
 
{Giả sử rằng vector trung bình $\bar{\mathbf{x}} = \mathbf{0}$}. Khi đó, \eqref{eqn:28_3} tương đương với
\begin{equation} 
\label{eqn:28_4}
\bX \approx \tilde{\bX} = \bU_K \mathbf{Z}
\end{equation} 
Bài toán tối ưu của PCA sẽ trở thành: 
\begin{equation} 
\label{eqn:28_5}
\begin{aligned} 
  \bU_K, \mathbf{Z} &=& \arg \min_{\bU_K, \mathbf{Z} } \|\bX - \bU_K
  \mathbf{Z}\|_F\\\ 
  \text{thoả mãn:}&& \bU_K^T \bU_K = \mathbf{I}_K & 
\end{aligned} 
\end{equation} 
với $\mathbf{I}_K \in \mathbb{R}^{K\times K}$ là ma trận đơn vị trong không gian $K$ chiều và điều kiện ràng buộc để đảm bảo các cột của $\bU_K$ tạo thành một hệ trực chuẩn. 
 
 
\subsection{Quan hệ giữa giữa hai phương pháp}
Bạn có nhận ra điểm tương đồng giữa hai bài toán tối ưu \eqref{eqn:28_1} và \eqref{eqn:28_5} với nghiệm của bài toán đầu tiên được cho trong \eqref{eqn:28_2}? Có thể nhận ra nghiệm của bài toán \eqref{eqn:28_5} chính là
\begin{align*} 
  \bU_K \quad \text{trong}\quad~\eqref{eqn:28_5} &= \bU_K \quad\text{trong} \quad
  \eqref{eqn:28_2} \\\ 
  \mathbf{Z} \quad\text{trong}\quad~\eqref{eqn:28_5} &= \mathbf{\Sigma}_K \mathbf{V}_K^T \quad\text{trong} \quad \eqref{eqn:28_2} 
\end{align*} 
Như vậy, nếu các điểm dữ liệu được biểu diễn bởi các cột của một ma trận và
trung bình các cột của ma trận đó là vector không thì nghiệm của bài toán PCA được rút ra trực tiếp từ SVD cắt ngọn của 
ma trận đó. Nói cách khác, việc đi tìm nghiệm cho PCA chính là việc giải một
bài toán phân tích ma trận thông qua SVD.
 
 
\section{Làm thế nào để chọn số chiều của dữ liệu mới}
 
Một câu hỏi được đặt ra là, làm thế nào để chọn giá trị $K$  --  chiều của dữ
liệu mới  --  với từng dữ liệu cụ thể? 
 
Thông thường, $K$ được chọn dựa trên việc {lượng thông tin muốn giữ lại}. Ở đây, toàn bộ thông tin chính là tổng phương sai của toàn bộ các chiều dữ liệu. Lượng dữ liệu muốn dữ lại là tổng phương sai của dữ liệu trong hệ trục toạ độ mới. 
 
 Nhắc lại rằng trong mọi hệ trục toạ độ, tổng phương sai của dữ liệu là như nhau
 và bằng tổng các trị riêng của ma trận hiệp phương sai $\sum_{i=1}^D
 \lambda_i$. Thêm nữa, PCA giúp giữ lại lượng thông tin (tổng các phương sai) là
 $\sum_{i=1}^K \lambda_i$. Vậy ta có thể coi biểu thức:
\begin{equation} 
 \label{eqn:28_6}
   r_K = \frac{\sum_{i=1}^K \lambda_i}{\sum_{j=1}^D \lambda_j} 
\end{equation} 
là tỉ lệ thông tin được giữ lại khi số chiều dữ liệu mới sau PCA là $K$. Như
 vậy, giả sử ta muốn giữ lại 99\% dữ liệu, ta chỉ cần chọn $K$ là số tự nhiên
 nhỏ nhất sao cho $r_K \geq 0.99$.
 
Khi dữ liệu phân bố quanh một không gian con, các giá trị phương sai lớn nhất
ứng với các $\lambda_i$ đầu tiên cao gấp nhiều lần các phương sai còn lại.
Khi đó, ta có thể chọn được $K$ khá nhỏ để đạt được $r_K \geq 0.99$.
 
 
\section{Lưu ý về tính toán phân tích thành phần chính}
Có hai trường hợp trong thực tế mà chúng ta cần lưu ý về PCA. Trường hợp thứ
nhất là lượng dữ liệu có được nhỏ hơn rất nhiều so với số chiều dữ liệu. Trường
hợp thứ hai là khi lượng dữ liệu trong tập huấn luyện rất lớn, việc tính toán ma trận hiệp phương sai và trị riêng đôi khi trở nên
bất khả thi. Có những hướng giải quyết hiệu quả cho các trường hợp này.
 
{Trong mục này, ta sẽ coi như dữ liệu đã được chuẩn hoá, tức đã được trừ đi
vector kỳ vọng. Khi đó, ma trận hiệp phương sai sẽ là $\mathbf{S} =
\frac{1}{N}\bX\bX^T$.} 
 
\subsection{Số chiều dữ liệu nhiều hơn số điểm dữ liệu}
 
Đó là trường hợp $D > N$, tức ma trận dữ liệu $\bX$ là một \textit{ma trận cao}.
Khi đó, số trị riêng khác không của ma trận hiệp phương sai $\mathbf{S}$ sẽ
không vượt quá hạng của nó, tức không vượt quá $N$. Vậy ta cần chọn $K \leq N$
vì không thể chọn ra được nhiều hơn $N$ trị riêng khác không của một ma trận có hạng
bằng $N$.
 
Việc tính toán các trị riêng và vector riêng cũng có thể được thực hiện một cách hiệu quả dựa trên các tính chất sau đây: 
\begin{enumerate}
    \item  Trị riêng của $\mathbf{A}$ cũng là trị riêng của $k\mathbf{A}$ với $k \neq 0$ bất kỳ. Điều này có thể được suy ra trực tiếp từ định nghĩa của trị riêng và vector riêng. 
 

\item Trị riêng của $\mathbf{AB}$ cũng là trị riêng của
$\mathbf{BA}$ với $\mathbf{A} \in \mathbb{R}^{d_1 \times d_2}, \mathbf{B} \in \mathbb{R} ^{d_2 \times d_1}$ là các ma trận bất kỳ và $d_1, d_2$ là các số tự nhiên khác không bất kỳ. 
 
Như vậy, thay vì tìm trị riêng của ma trận hiệp phương sai $\mathbf{S} \in \mathbb{R}^{D\times D}$, ta đi tìm trị riêng của ma trận $\mathbf{T} = \bX^T \bX \in \mathbb{R}^{N \times N}$ có số chiều nhỏ hơn (vì $N < D$). 
 
\item Nếu $(\lambda, \mathbf{u})$ là một cặp trị riêng,
vector riêng của $\mathbf{T}$ thì $(\lambda, \mathbf{Xu})$ là một cặp trị
riêng, vector riêng của $\mathbf{S}$. Thật vậy:
\begin{eqnarray} 
  \label{eqn:28_8}
  \bX^T \mathbf{Xu} = \bT\bu = \lambda \mathbf{u} 
  \Rightarrow (\bX\bX^T)(\mathbf{Xu}) = \lambda (\mathbf{Xu} ) 
\end{eqnarray} 
Dấu bằng thứ nhất xảy ra theo định nghĩa của trị riêng và vector riêng. 
\end{enumerate}
 
Như vậy, ta có thể hoàn toàn tính được trị riêng và vector riêng của ma trận
hiệp phương sai $\bS$ dựa trên một ma trận $\bT$ có kích thước nhỏ hơn. Việc
này trong nhiều trường hợp khiến thời gian tính toán giảm đi đáng kể. 
 
% \subsubsection{Chuẩn hoá các vector riêng}
 
% \textit{Nhắc lại định nghĩa không gian riêng: Không gian riêng ứng với trị riêng của một ma trận là không gian sinh (span subspace) tạo bởi toàn bộ các vector riêng ứng với trị riêng đó.} 
 
% Việc cuối cùng phải làm là chuẩn hoá các vector riêng tìm được sao cho chúng tạo thành một hệ trực chuẩn. Việc này có thể dựa trên hai điểm sau đây: 
 
% \textbf{Thứ nhất}, nếu $\mathbf{A}$ là một ma trận đối xứng, $(\lambda_1,
% \mathbf{x}_1), (\lambda_2, \mathbf{x}_2)$ là các cặp (trị riêng, vector riêng)
% của $\mathbf{A}$ với $\lambda_1 \neq \lambda_2$, thì
% $\mathbf{x}_1^T\mathbf{x}_2 = 0$. Nói cách khác, hai vector bất kỳ trong hai
% không gian riêng khác nhau của một ma trận đối xứng trực giao với nhau.
% Chứng mình: 
% \begin{eqnarray} 
%   \mathbf{x}_2^T \mathbf{Ax}_1 = \mathbf{x}_1^T \mathbf{Ax}_2 = \lambda_1 \mathbf{x}_2^T \mathbf{x}_1 = \lambda_2 \mathbf{x}_1^T \mathbf{x}_2 \Rightarrow \mathbf{x}_1^T \mathbf{x}_2 = 0 
% \end{eqnarray} 
% Dấu bằng cuối cùng xảy ra vì $\lambda_1 \neq \lambda_2$. 
 
% \textbf{Thứ hai}, với các trị riêng độc lập tìm được trong một không gian riêng,
% ta có thể dùng quá trình trực giao hoá Gram-Schmit để chuẩn hoá chúng về một hệ
% trực
% chuẩn.
 
% Kết hợp hai điểm trên, ta có thể thu được các vector riêng tạo thành một hệ trực chuẩn, chính là ma trận $\bU_K$ trong PCA. 
 
 
\subsection{Với các bài toán quy mô lớn}
Trong rất nhiều bài toán quy mô lớn, ma trận hiệp phương sai là một ma trận rất lớn. Ví dụ, có một triệu bức ảnh
1000 $\times$ 1000 pixel, như vậy $D = N = 10^6$ là các số rất lớn, việc trực
tiếp tính toán trị riêng và vector riêng cho ma trận hiệp phương sai là không
khả thi. Lúc này, các trị riêng và vector riêng của ma trận hiệp phương sai thường được tính thông qua \textit{power method}
(\url{https://goo.gl/eBRPxH}). 

% Phương pháp này nói rằng, nếu thực hiện quy trình sau, ta sẽ tìm được cặp trị riêng và vector đầu tiên của một ma trận nửa xác định dương: 
% % <hr> 
% \textbf{Phương pháp Power tìm trị riêng và vector riêng của một ma trận nửa xác định dương $\mathbf{A} \in \mathbb{R}^{n \times n}$}: 
% \begin{enumerate}
%   \item Chọn một vector $\mathbf{q}^{(0)} \in \mathbb{R}^n, \|\mathbf{q}^{(0)}\|_2 = 1$ bất kỳ. 
%   \item Với $k = 1, 2, \dots$, tính: $\mathbf{z} = \mathbf{Aq}^{(k-1)}$. 
%   \item Chuẩn hoá: $\mathbf{q}^{(k)} = \mathbf{z} / \|\mathbf{z}\|_2$. 
%   \item Nếu $\|\mathbf{q}^{(k)} - \mathbf{q}^{(k-1)}\|_2$ đủ nhỏ thì dừng lại. Nếu không, $k := k + 1$ rồi quay lại Bước 2. 
%   \item $\mathbf{q}^{(k)}$ chính là vector riêng ứng với trị riêng lớn nhất $\lambda_1 = (\mathbf{q}^{(k)})^T\mathbf{A}\mathbf{q}^{(k)}$. 
% \end{enumerate}
% % <hr> 
 
% Quy trình này hội tụ khá nhanh và đã được chứng minh \href{http://www.cs.huji.ac.il/~csip/tirgul2.pdf}{tại đây}. Phần chứng minh tương đối đơn giản và không mang lại nhiều thông tin hữu ích, tôi xin được bỏ qua. 
 
% Để tìm vector riêng và trị riêng thứ hai của ma trận $\mathbf{A}$, chúng ta dựa trên định lý sau: 

% \textbf{Định lý:} Nếu ma trận nửa xác định dương $\mathbf{A}$ có các trị riêng $\lambda_1 \geq \lambda_2 \geq \dots \geq \lambda_n ( \geq 0)$ và các vector riêng tương ứng $\mathbf{v}_1, \dots, \mathbf{v}_n$, hơn nữa các vector riêng này tạo thành 1 hệ trực chuẩn, thì ma trận: 
% \begin{equation} 
%   \mathbf{B} = \mathbf{A} - \lambda_1 \mathbf{v}_1 \mathbf{v}_1^T 
% \end{equation} 
% có các trị riêng $\lambda_2 \geq \lambda_3 \geq \dots \geq \lambda_n \geq 0$ và các vector riêng tương ứng là $\mathbf{v}_2, \mathbf{v}_3, \dots, \mathbf{v}_n, \mathbf{v}_1$. 

% \textit{Chứng minh: }
 
% Với $i = 1$: 
% \begin{eqnarray} 
%   \mathbf{Bv}_1 &=& (\mathbf{A} - \lambda_1 \mathbf{v}_1 \mathbf{v}_1^T) \mathbf{v} \\\
%   &= & \mathbf{Av}_1 - \lambda_1 \mathbf{v}_1 = \mathbf{0} \\\ 
% \end{eqnarray} 
 
% Với $i > 1$: 
% \begin{eqnarray} 
%   \mathbf{Bv}_i &=& (\mathbf{A} - \lambda_1 \mathbf{v}_1 \mathbf{v}_1^T)\mathbf{v}_i \\\ 
%   &=& \mathbf{Av}_i - \lambda_1 \mathbf{v}_1 (\mathbf{v}_1^T \mathbf{v}_i) \\\ 
%   &=& \mathbf{Av}_i = \lambda_i \mathbf{v}_i 
% \end{eqnarray} 
 
% Như vậy định lý đã được chứng minh $\square$. 
 
% Lúc này, $(\lambda_2, \mathbf{v}_2)$ lại trở thành cặp trị riêng-vector riêng lớn nhất của $\mathbf{B}$. Cách tìm hai biến số này một lần nữa được thực hiện bằng Phương pháp Power. 
 
% Tiếp tục quy trình này, ta sẽ tìm được (xấp xỉ) tất cả các trị riêng và vector riêng tương ứng của ma trận hiệp phương sai. Cũng xin lưu ý rằng ta chỉ cần tìm tới trị riêng thứ $K$ của ma trận hiệp phương sai. Cách làm này trên thực tế được sử dụng rất nhiều. 
 
% Phương pháp Power còn là thuật toán cơ bản trong \href{https://en.wikipedia.org/wiki/PageRank}{Google PageRank} giúp sắp xếp các website theo mức độ phổ biến giảm dần. PageRank chính là nền móng của Google; ngày nay, việc tìm kiếm trong Google sử dụng nhiều thuật toán nâng cao hơn PageRank. Tôi sẽ có một bài riêng về Google PageRank sau khi nói về Chuỗi Markov và Mô hình Markov ẩn. 
 
 
\section{Một số ứng dụng}
 Ứng dụng đầu tiên của PCA chính là việc giảm chiều dữ liệu, giúp
 việc lưu trữ và tính toán được thuận tiện hơn. Thực tế cho thấy, nhiều khi làm
 việc trên dữ liệu đã được giảm chiều mang lại kết quả tốt hơn so với dữ liệu
 gốc. Thứ nhất, có thể phần dữ liệu mang thông tin nhỏ bị lược đi chính là phần
 gây nhiễu, những thông tin quan trọng hơn đã được giữ lại. Thứ hai, số điểm dữ
 liệu nhiều khi ít hơn số chiều dữ liệu. Khi có quá ít dữ liệu và số chiều dữ
 liệu quá lớn, quá khớp rất dễ xảy ra. Việc giảm chiều dữ liệu phần nào giúp
 khắc phục hiện tượng này. 

 Dưới đây là hai ví dụ về ứng dụng của PCA trong bài toán phân loại khuôn mặt và dò điểm bất thường. 
 
\subsection{Khuôn mặt riêng}
\index{khuôn mặt riêng -- eigenface}
\textit{Khuôn mặt riêng} (eigenface) từng là một trong những kỹ thuật phổ biến
trong bài toán nhận dạng khuôn mặt. Ý tưởng của khuôn mặt riêng là đi tìm một
không gian có số chiều nhỏ hơn để mô tả mỗi khuôn mặt, từ đó sử dụng vector
trong không gian thấp chiều này như vector đặc trưng cho bộ phân loại. Điều đáng
nói là một bức ảnh khuôn mặt có kích thước khoảng 200 $\times$ 200 sẽ có số
chiều là 40k -- một số rất lớn, trong khi đó, vector đặc trưng thường chỉ có số
chiều bằng vài trăm hoặc vài nghìn. Khuôn mặt riêng thực ra chính là PCA. Các
khuôn mặt riêng chính là các vector riêng ứng với những trị riêng lớn nhất của
ma trận hiệp phương sai.

\index{cơ sở dữ liệu khuôn mặt Yale -- Yale face database}
Trong phần này, chúng ta làm một thí nghiệm nhỏ trên \textit{cơ sở dữ liệu
khuôn mặt Yale} (\url{https://goo.gl/LNg8LS}). Các bức ảnh trong thí nghiệm
này đã được căn chỉnh cho cùng với kích thước và khuôn mặt nằm trọn vẹn trong
một hình chữ nhật có kích thước $116 \times  98$ điểm ảnh. Có tất cả 15 người khác
nhau, mỗi người có 11 bức ảnh được chụp ở các điều kiện ánh sáng và cảm xúc khác
nhau, bao gồm \pythoninline{'centerlight', 'glasses', 'happy', 'leftlight',
'noglasses', 'normal', 'rightlight','sad', 'sleepy', 'surprised'} và
\pythoninline{'wink'}. Hình \ref{fig:28_1} minh hoạ các bức ảnh của
người có id là 10. 
% <hr> 
% <div class="imgcap"> 
% <img src ="/assets/28_pca2/yaleb_exs.png" align = "center" width = "800"> 
% </div> 
 
% <div class = "thecap" align = "left">Hình 1: Ví dụ về ảnh của một người trong Yale Face Database. </div> 
% <hr> 

\begin{figure}[t]
\centering
    \includegraphics[width = \textwidth]{Chapters/07_DimemsionalityReduction/28_pca2/latex/yaleb_exs.pdf}
    \caption[]{Ví dụ về ảnh của một người trong Yale Face Database.}
    \label{fig:28_1}
\end{figure}
Ta thấy rằng số chiều dữ liệu $116 \times 98 = 11368$ là một số khá
lớn. Tuy nhiên, vì chỉ có tổng cộng $15 \times 11 = 165$ bức ảnh nên ta có thể
nén các bức ảnh này về dữ liệu mới có chiều nhỏ hơn 165. Trong ví dụ này, chúng
ta chọn $K = 100$.
 
Dưới đây là đoạn code thực hiện PCA cho toàn bộ dữ liệu. Ở đây, A trong \pythoninline{sklearn} được sử dụng:
\newpage  
\begin{lstlisting}[language=Python]
import numpy as np 
from scipy import misc                     # for loading image 
np.random.seed(1) 
 
# filename structure 
path = 'unpadded/' # path to the database 
ids = range(1, 16) # 15 persons 
states = ['centerlight', 'glasses', 'happy', 'leftlight', 
          'noglasses', 'normal', 'rightlight','sad', 
          'sleepy', 'surprised', 'wink' ] 
prefix = 'subject' 
surfix = '.pgm' 
# data dimension 
h, w, K = 116, 98, 100 # hight, weight, new dim 
D = h * w 
N = len(states)*15 
# collect all data 
X = np.zeros((D, N)) 
cnt = 0 
for person_id in range(1, 16): 
    for state in states: 
        fn = path + prefix + str(person_id).zfill(2) + '.' + state + surfix 
        X[:, cnt] = misc.imread(fn).reshape(D) 
        cnt += 1 
 
# Doing PCA, note that each row is a datapoint 
from sklearn.decomposition import PCA 
pca = PCA(n_components=K) # K = 100 
pca.fit(X.T) 
# projection matrix 
U = pca.components_.T 
\end{lstlisting}
 
% Chú ý rằng các hàm của \pythoninline{sklearn} đều chọn dữ liệu ở dạng hàng. Còn tôi thường chọn dữ liệu ở dạng cột cho thuận tiện trong biểu diễn toán học. Trước khi sử dụng \pythoninline{sklearn}, bạn đọc chú ý chuyển vị ma trận dữ liệu. 
 
Trong dòng \pythoninline{pca = PCA(n_components=K)}, nếu
\pythoninline{n_components} là một số thực trong khoảng $(0, 1)$, PCA sẽ thực
hiện việc tìm $K$ dựa trên biểu thức~\eqref{eqn:28_6}. 
 
% ******************************************************************************
\begin{figure}[t]
\centering
    \includegraphics[width = \textwidth]{Chapters/07_DimemsionalityReduction/28_pca2/latex/yaleb_eig.pdf}
    \caption[]{Các eigenfaces tìm được bằng PCA.}
    \label{fig:28_2}
\end{figure}
% ******************************************************************************
 
Hình \ref{fig:28_2} biểu diễn 18 vector riêng đầu tiên (18 cột đầu tiên của
$\bU_k$) tìm được bằng PCA. Các vector đã được \pythoninline{reshape} về cùng
kích thước như các bức ảnh gốc. Nhận thấy các
vector thu được ít nhiều mang thông tin của mặt người. Thực tế, một khuôn mặt
gốc sẽ được xấp xỉ như tổng có trọng số của các {khuôn mặt} này. Vì các
vector riêng này đóng vai trò như cơ sở của không gian mới với ít chiều hơn,
chúng còn được gọi là \textit{khuôn mặt riêng} hoặc \textit{khuôn mặt chính}. Từ \textit{chính} được dùng vì nó đi kèm với văn cảnh
của \textit{phân tích thành phần chính}.
% ****************************************************************************** 
\begin{figure}[t]
\centering
    \includegraphics[width = \textwidth]{Chapters/07_DimemsionalityReduction/28_pca2/latex/yaleb_ori_res.pdf}
    \caption[]{Hàng trên: các ảnh gốc. Hàng dưới: các ảnh được tái tạo dùng khuôn mặt riêng. Ảnh ở hàng dưới có nhiễu nhưng vẫn mang những đặc điểm riêng mà mắt người có thể phân biệt được.}
    \label{fig:28_3}
\end{figure}
% ****************************************************************************** 

Để xem mức độ hiệu quả của phương pháp này, chúng ta  minh hoạ các bức ảnh gốc và các bức ảnh được xấp xỉ bằng PCA như trên
Hình~\ref{fig:28_3}. Các khuôn mặt nhận được vẫn mang khá đầy đủ thông tin của
các khuôn mặt gốc. Đáng chú ý hơn, các khuôn mặt trong hàng dưới được suy ra
từ một vector 100 chiều, so với 11368 chiều như ở hàng trên. 
 



% Phần còn lại của source code có thể được tìm thấy \href{https://github.com/tiepvupsu/tiepvupsu.github.io/blob/master/assets/28_pca2/python/EigenFaces.ipynb}{tại đây}. 
 
 
\subsection{Dò tìm điểm bất thường}
Ngoài các ứng dụng về nén và phân loại, PCA còn được sử dụng trong nhiều lĩnh
vực khác. \textit{Dò tìm điểm bất thường} (abnormal detection hoặc {outlier
detection}) là một trong số đó~\cite{shyu2003novel,lakhina2004diagnosing}.
 
Ý tưởng cơ bản là giả sử tồn tại một không gian con mà các sự kiện bình thường
nằm gần trong khi các sự kiện bất thường nằm xa không gian con đó. Hơn nữa, số
sự kiện bất thường có một tỉ lệ nhỏ. Như vậy, PCA có thể được sử dụng trên toàn
bộ dữ liệu để tìm ra các thành phần chính, từ đó suy ra không gian con mà các điểm bình thường nằm gần.
Việc xác định một điểm là bình thường hay bất thường được xác định bằng cách đo
khoảng cách từ điểm đó tới không gian con tìm được. Hình~\ref{fig:28_4} minh hoạ
cho việc xác định các sự kiện bất thường bằng PCA. 
 
\begin{figure}[t]
    % caption on side     
   
    \floatbox[{\capbeside\thisfloatsetup{capbesideposition={right,top},capbesidewidth=6.5cm}}]{figure}[\FBwidth]
    {\caption{ PCA cho bài toán dò tìm điểm bất thường. Giả sử
    các sự kiện {bình thường} chiếm đa số và nằm gần  một không
    gian con nào đó. Khi đó, nếu làm PCA trên toàn bộ dữ liệu, không gian con
    thu được gần với không gian con của tập các sự kiện {bình thường}.
    Lúc này, các
    điểm hình tròn to đậm hơn có thể được coi là các sự kiện {bất thường} vì chúng nằm xa không gian con chính.} 
    \label{fig:28_4}}
    { % figure here
    \includegraphics[width=.45\textwidth]{Chapters/07_DimemsionalityReduction/28_pca2/latex/abnormal.pdf}
    }
\end{figure}

 
\section{Thảo luận}
\begin{itemize}
\item PCA là phương pháp giảm chiều dữ liệu dựa trên việc tối đa lượng
thông tin được giữ lại. Lượng thông tin được giữ lại được đo bằng tổng các
phương sai trên mỗi thành phần của dữ liệu. Lượng dữ liệu sẽ được giữ lại nhiều
nhất khi các chiều dữ liệu còn lại tương ứng với các vector riêng của trị riêng
lớn nhất của ma trận hiệp phương sai.

\item Với các bài toán quy mô lớn, đôi khi việc tính toán trên toàn bộ dữ liệu
là không khả thi vì vấn đề bộ nhớ. Giải pháp là thực hiện PCA lần đầu
trên một tập con dữ liệu vừa với bộ nhớ, sau đó lấy một tập con khác để {từ từ} (\textit{incrementally}) cập nhật nghiệm của PCA tới khi hội
tụ. Ý tưởng này khá giống với mini-batch gradient descent và được gọi là
incremental PCA~\cite{zhao2006novel}.
 
\item Ngoài ra, còn rất nhiều hướng mở rộng của PCA, bạn đọc có thể tìm kiếm
theo từ khoá: Sparse PCA~\cite{d2005direct}, Kernel PCA~\cite{mika1999kernel},
Robust PCA~\cite{candes2011robust}. 

\item Mã nguồn trong chương này có thể được tìm thấy tại
\url{https://goo.gl/zQ3DSZ}. 
\end{itemize}
 